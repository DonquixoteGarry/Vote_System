% -*- coding: utf-8 -*-


\begin{zhaiyao}
\begin{spacing}{1.5}
{


在神经网络模型训练中,对植入相关触发器后门的样本和模型的分析一直是重要的领域.对于识别模型的安全性和识别样本的恶意性,是其中重要的研究方向.

本次选题基于两个假设:恶意训练集中恶意样本占比不高,以及人具有手动分辨少量恶意样本的能力.

本次选题尝试使用目标训练集的等规模的随机子集,并以其训练若干模型,进行样本的投票式标签预测,以筛查出恶意样本.结合人工识别,评估恶意样本的影响.

因此,本次选题的重点在于降低恶意样本的漏报率,并以漏报率作为该衡量该评估方式的有效性的指标.

GitHub repo: \url{https://github.com/DonquixoteGarry/test}
}
\end{spacing}
\end{zhaiyao}

\begin{guanjianci}
投票机制;后门检测
\end{guanjianci}



\begin{abstract}
\begin{spacing}{1.5}

While  Training

\end{spacing}
\end{abstract}


\begin{keywords}
voting mechanism;
\end{keywords} 